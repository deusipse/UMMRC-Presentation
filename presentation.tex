\documentclass[xcolor={usenames,dvipsnames}]{beamer}

\usetheme{Dresden}
\usecolortheme{beaver}

\usepackage{import}
\usepackage{xifthen}
\usepackage{pdfpages}
\usepackage{transparent}
\usepackage{caption}
\usepackage{comment}

\newcommand{\incfig}[1]{%
  \def\svgwidth{0.3\textwidth}
  \import{./figures/}{#1.pdf_tex}
}


% https://tex.stackexchange.com/questions/137022/how-to-insert-page-number-in-beamer-navigation-symbols
\setbeamertemplate{footline}[frame number]
\expandafter\def\expandafter\insertshorttitle\expandafter{%
  \insertshorttitle\hfill%
  \insertframenumber\,/\,\inserttotalframenumber}
\setbeamercolor{institute in head/foot}{parent=palette primary}

\usepackage{mathtools, amssymb, amsthm, parskip}

\usepackage[inline]{asymptote}
\def\asydir{asy}

\usepackage{microtype}
\usepackage{multicol}

\newtheorem{claim}{Claim}[theorem]

\title{Question 8}
\subtitle{Mathematics and Statistics Research Competition}
\author{Jiamu Li \& Frank Tang \& Edward Wang}
\institute{Scotch College}
\date{October 24, 2022}

\DeclareMathOperator{\prob}{Pr}
\DeclarePairedDelimiter\floor{\lfloor}{\rfloor}

\begin{document}

\begin{frame}
  \titlepage
\end{frame}


\section{The Question}
\begin{frame}{The Situation}
  A particle generator emits X or Y particles into an empty tube, with equal probability. Shots are independent.
  \vspace{1cm}
  \begin{center}
   \begin{figure}[ht]
    \centering
    \def\svgwidth{0.8\textwidth}
    \import{./figures/}{drawing.pdf_tex}
  \end{figure}
  \end{center}
\end{frame}
\section{Problem 1}
\begin{frame}{Problem 1}
\begin{itemize}
  \item What is the probability that no two X-particles are next to each other?
\end{itemize}

\begin{equation*}
  \prob(\text{No consec X-particles}) = \frac{\# \text{Arrangements w/o consec X-particles}}{\# \text{Total arrangements}}
\end{equation*}

\end{frame}

\begin{frame}
  Let $g(n)$ denote the number of arrangements without two touching X particles after  $n $ shots.
  
  Consider the first particle, which is either X or Y.
  \begin{itemize}
    \item If the first particle is Y, then it doesn't affect the number of arrangements giving us $g(n-1)$.
    \item Otherwise, the first particle is X and hence the next particle must be Y. Hence there are  $g(n-2)$ arrangements.
  \end{itemize}
\end{frame}
\begin{frame}
  Adding these two up, we get  \[
    g(n) = g(n-1) + g(n-2)
  .\] This satisfies the Fibonacci recursion, and since $g(1) = 2$ and $g(2) = 3$ we get $g(n) = F_{n+2}$, where $F_n$ is the $n$th Fibonacci number.

  Since the total number of arrangements of $n$ particles is $2^{n}$, since each particle is either X or Y, the probability that no two X particles are consecutive is \[
    \frac{F_{n+2}}{2^n}
  .\] 
\end{frame}

\begin{comment}
\begin{frame}

  Consider a tube with $n$ particles, $k$ of them are X, $n-k$ are Y.
  \begin{equation*}
    \underbrace{\text{YY}\dots\text{YY}}_{n-k}
  \end{equation*}
  Placing all $k$ of the X particles in the $n-k+1$ gaps will ensure no consecutive X's. Hence, the number of arrangements without consecutive X's is
  \begin{equation*}
    \sum_{k=0}^n \binom{n-k+1}{k}.
  \end{equation*}

\end{frame}

\begin{frame}[fragile]

\begin{multicols}{2}
  \begin{center}
    \begin{asy}
      unitsize(0.38cm);
      defaultpen(fontsize(9pt));

      int fib(int n) {
        if (n == 0) 
        return 0;
        else if (n == 1) 
        return 1;
        int fibs[] = {0, 1};
        while (fibs.length <= n) {
          fibs.push(fibs[fibs.length - 1] + fibs[fibs.length - 2]);
        }
        return fibs[fibs.length - 1];
      }

      int rows = 6;

      for (int i = 0; i <= rows; ++i) {
        pair A = (-i, -i*sqrt(3));
        pair B = (3, sqrt(3));
        path idk = A -- (A + 0.5*i*B) + 0.5*B;
        draw(L = Label(string(fib(i+1)), position = EndPoint), idk, blue+opacity(0.4));
      }

      for (int i = 0; i < floor(rows/2)+1; ++i) {
        pair A = (-rows, -rows*sqrt(3));
        pair B = (A + i*(3, sqrt(3)));
        if (i != 0)
        draw(B + 0.25*(-1, sqrt(3)) -- B + 0.75*(-1, sqrt(3)), red);
        if (i != floor(rows/2))
        draw(B + 0.25*(1, sqrt(3)) -- B + 0.75*(1, sqrt(3)), red);
      }

      for (int i = 0; i <= rows; ++i) {
        for (int j = 0; j <= i; ++j) {
          label(string(choose(i, j)), (j*2 - i, -i*sqrt(3)));
        }
      }
    \end{asy}
  \end{center}
  \columnbreak
  Observe the diagram of Pascal's triangle.

  The formula represents the sum of each diagonal, which is made up of the sum of the previous two diagonals, satisfying the Fibonacci recursion. 

  Hence,   \[\sum_{k=0}^n \binom{n-k+1}{k} = F_{n+2}.\]
\end{multicols}

\end{frame}

\begin{frame}
  Since each of the $n$ particles is X or Y, the number of arrangements is $2^n$.

  Hence the probability that no two X-particles are consecutive after $n$ shots is \[
    \frac{F_{n+2}}{2^n}
  .\] 
\end{frame}
\end{comment}

\section{Problem 2}
\begin{frame}{Problem 2}
  Two consecutive X particles now collapse into one. 
  \begin{itemize}
    \item Find the average number of particles after $n$ shots.
  \end{itemize}

  This problem can be solved as a subcase of the more general following problem.
\end{frame}

\section{Problem 3}
\begin{frame}{Problem 3}
  Two touching X particles now collapse into 1.

  Continuing from the previous problem, let the probability of firing an X particle be some  $p \in (0, 1)$.
  Suppose at each shot, the probability of firing an X particle is  some $p \in (0, 1)$.
  \begin{itemize}
    \item Will the proportion of X particles in the tube stabilise as the number of shots goes to infinity?
    \item If so, is there a formula for this number?
  \end{itemize}
  We have:
  \begin{equation*}
    \text{Proportion} = \frac{\#\text{X particles}}{\#\text{Particles}} = \frac{\#\text{Particles - Y particles}}{\#\text{Particles}}
  \end{equation*}
\end{frame}

\begin{frame}
  The total number of particles stays the same if an X particle hits an X particle.

  This happens with probability $p^2$.

  \begin{multicols}{2}
  \begin{figure}[ht]
    \centering
    \incfig{something}
    \label{fig:stuff}
  \end{figure}
  \columnbreak
  Otherwise, the number of particles increases by 1 with probability $1-p^2$. So, \[
    T_{n+1} = T_n + 1-p^2
  .\] Solving, we get \[
    T_n = (1-p^2)n + p^2
    .\] \end{multicols}
    Setting $p = \frac{1}{2}$ yields the solution for Problem 2: $
    T_n = \frac{3}{4}n + \frac{1}{4}
  $.
\end{frame}
\begin{frame}
  After $n$ shots, clearly the number of Y particles is \[
    (1-p)n
  .\] Hence, the proportion is 
  \begin{equation*}
    \text{Proportion} = \frac{(1-p^2)n + p^2 - (1-p)n}{(1-p^2)n + p^2}.
\end{equation*}
As $n$ approaches infinity, the proportion approaches \[
  \frac{p}{1+p}
.\] 
\end{frame}

\section{Generalisation}
\begin{frame}{Generalisation}
  \begin{itemize}
    \item What happens if $m$ consecutive X particles collapsed into $n$ particles?
    \item Everything else remains the same, i.e. Y particles don't collapse, probability of X particle is $p \in (0, 1)$
    \item What is the average number of particles after $k$ shots?
  \end{itemize}
\end{frame}

\begin{frame}{Recursion}
  If an X particles hits $m-1$ consecutive X particles, then they collapse into $n$ X particles, decreasing the number of particles by $m-1-n$.

  Otherwise, the number of particles increases by 1.

  Let the probability of having  $m-1$ consecutive X particles be $\vartheta$. 

  \begin{align*}
    T_{k+1} &= T_k + 1-p\vartheta + p\vartheta(n-m+1) \\
            &= T_k + 1 - p\vartheta(m - n)
  \end{align*}
\end{frame}
\begin{frame}{Calculating $\vartheta$}
  $\vartheta$ is the probability that there are $m-1$ X particles in a row.

  The string of X particles must start with either a Y unless it is the beginning of the sequence. 
  % \begin{figure}[H]
  %   \vspace{-1.5em}
  %   \begin{align*}
  %       &\underbrace{\text{X}\dots\text{X}}_{m-1}\text{Y} \tag{A}\\
  %       &\underbrace{\text{X}\dots\text{X}}_{m-1} \tag{B}
  %   \end{align*}
  %   \vspace{-2em}
  % \end{figure}
  \vspace{1cm}
\begin{center}
  \begin{figure}[ht]
    \centering
    \def\svgwidth{0.6\textwidth}
    \import{./figures/}{a.pdf_tex}
\end{figure}
  \begin{figure}[ht]
    \centering
    \def\svgwidth{0.6\textwidth}
    \import{./figures/}{b.pdf_tex}
  \end{figure}
\end{center}

\end{frame}
\begin{frame}{Configuration A}
  %   \begin{figure}[H]
  %   \vspace{-1.5em}
  %   \begin{align*}
  %       &\underbrace{\text{X}\dots\text{X}}_{m-1}\text{Y} \tag{A}
  %   \end{align*}
  %   \vspace{-2em}
  % \end{figure}
  \begin{center}
  \begin{figure}[ht]
    \centering
    \def\svgwidth{0.6\textwidth}
    \import{./figures/}{a.pdf_tex}
  \end{figure}
\end{center}

  If the sequence has a Y particle at the end, then the probability is simply $(1-p)p^{m-1}$.

  But that doesn't account for previous collapses.
  \begin{figure}[H]
      \vspace{-1.5em}
      \begin{equation*}
        \underbrace{\text{X}\dots\text{X}}_{m + (m-n) - 1}\text{Y} \quad \longrightarrow \quad \underbrace{\text{X}\dots\text{X}}_{n+(m-n)-1}\text{Y}\quad = \quad \underbrace{\text{X}\dots\text{X}}_{m-1}\text{Y} 
      \end{equation*}
      \vspace{-2em}
    \end{figure}
  \end{frame}
  \begin{frame}{Configuration A}
    Adding $m-n$ particles to $m-1$ reverts it back to $m-1$, which can happen $\floor*{\dfrac{k-m}{m-n}}$ times. Summing this up gives \[\sum_{a = 0}^{\floor*{\frac{k-m}{m-n}}} (1-p)p^{a(m-n) + m-1}.\]
\end{frame}
\begin{frame}{Configuration B}
  %   \begin{figure}[H]
  %   \vspace{-1.5em}
  %   \begin{align*}
  %       &\underbrace{\text{X}\dots\text{X}}_{m-1} \tag{B}
  %   \end{align*}
  %   \vspace{-2em}
  % \end{figure}
    \begin{center}
  \begin{figure}[ht]
    \centering
    \def\svgwidth{0.6\textwidth}
    \import{./figures/}{b.pdf_tex}
  \end{figure}
\end{center}

  Accounting for previous collapses, this configuration can obviously happen with a probability of $p^{k}$, if $k = a(m-n) + m-1$ for some integer $a$. Hence, the probability is \[
    \varepsilon = \begin{cases} p^{k}, &k-m+1 \equiv 0 \bmod m-n \\ 0, &k-m+1 \not\equiv 0 \bmod m-n. \end{cases}
  .\] 
\end{frame}
\begin{frame}{Configuration B}
  Adding these two probabilities gives us \[
    \vartheta = \sum_{a=0}^{\floor*{\frac{k-m}{m-n}}} (1-p)p^{a(m-n)+m-1} + \varepsilon,
  \] where \[
    \varepsilon = \begin{cases} p^{k}, &k-m+1 \equiv 0 \bmod m-n \\ 0, &k-m+1 \not\equiv 0 \bmod m-n. \end{cases}
  \] 
\end{frame}

\begin{frame}{Putting it all together}
  Recall that we obtained a recursion for  $T_k$ earlier.
  \[T_{k+1} = T_k + 1 - p\vartheta(m - n)\]
  We simply sum the left hand side from $m-1$ to $k-1$ to get a closed expression:
  \begin{equation*}
  T_k = m-1 + \sum_{b=m-1}^{k-1} \left(1-p\vartheta(m-n)\right).
  \end{equation*}
  The formula works for $k \ge m$ since $T_k = k$ for  $k<m$.
\end{frame}

\begin{frame}
  We can even expand it, getting
  \[
  T_k = m-1+\sum _{b=m-1}^{k-1} \left(1-p \left(\sum _{a=0}^{\floor*{\frac{b-m}{m-n}} } (1-p) p^{a (m-n)+m-1}+\varepsilon\right) (m-n)\right)
  \] However, there is a notable case where this nasty formula simplifies dramatically.
\end{frame}

\begin{frame}{A special case}
  Let us consider what happens when $n = m-1 \iff m-n = 1$.
  \begin{equation*}
    T_k = m-1+\sum _{b=m-1}^{k-1} \left(1-p \left({\color{Orchid}\sum _{a=0}^{\floor*{\frac{b-m}{1}} } (1-p) p^{a (1)+m-1}+p^{b}}\right) (1)\right)
  \end{equation*}
  Even $\varepsilon$ becomes simplified into $p^{k}$ since $z \equiv 0 \bmod 1$ for any integer  $z$. Amazingly, we can simplify $\color{Orchid} \vartheta$ into  \[
    \vartheta = \sum _{a=0}^{k-m } (1-p) p^{a+m-1}+p^{k} = p^{m-1}
  \] since it is a geometric series. Now $\vartheta$ is constant as it no longer depends on  $k$.
\end{frame}
\begin{frame}
  Putting it back in, we get 
    \begin{align*}
      T_k &= m-1 + \sum_{b=m-1}^{k-1} (1-p({\color{Orchid} p^{m-1}})) \\
          &= m-1 + (k-m+1)(1-p^{m})
    \end{align*}
    Setting $m = 2, n = 1, p = \frac{1}{2}$, the parameters of Problem 3, we arrive at the exact same formula: \[
      T_k = (1 - p^2)k + p^2
    .\] 
\end{frame}

\section{Epilogue}
\begin{frame}{Use of computer simulation}
  We used C++ to quickly compute average number of particles after any number of shots.

  Computer simulations helped us determine that a formula existed when finding the pattern.

  They also helped us validate that our formula worked correctly.
\end{frame}
\begin{frame}{Extra research ideas}
  There is still extra things that can be explored.
  \begin{itemize}
    \item What if both X and Y can collapse?
    \item What if there are more than 2 particles?
    \item What if there are $n$ particles, $k$ of which can collapse?
  \end{itemize}
\end{frame}

\end{document}
