\documentclass{beamer}

\usetheme{Dresden}
\usecolortheme{beaver}

\usepackage{mathtools, amssymb, amsthm, parskip}
\newtheorem{claim}{Claim}[theorem]

\title{Question 8}
\subtitle{Mathematics and Statistics Research Competition}
\author{Jiamu Li \& Frank Tang \& Edward Wang}
\institute{Scotch College}
\date{October 24, 2022}

\DeclareMathOperator{\prob}{Pr}

\begin{document}

\begin{frame}
  \titlepage
\end{frame}

\begin{frame}
  \tableofcontents
\end{frame}

\section{The Question}
\begin{frame}{The Question}
  A particle generator is emitting two types of particles (called X and Y) into a long tube. The particles will line up in order after entering the tube. Initially, the tube is empty. At each shot, either an X- or Y-particle is randomly emitted into the tube with equal probability. Different shots are assumed to be independent from each other. Suppose that $n$ shots have been emitted.
\end{frame}
\section{Problem 1}
\begin{frame}{Problem 1}
\begin{itemize}
  \item What is the probability that no two X-particles are next to each other?
\end{itemize}

\begin{equation*}
  \prob(\text{No consecutive X-particles}) = \frac{\# \text{No consecutive X-particles}}{\# \text{Total arrangements}}
\end{equation*}

\end{frame}

\begin{frame}
\begin{claim}\label{combiformula}
  The number of arrangements with no consecutive X-particles is 
  \begin{equation}
    \sum_{k=0}^n \binom{n-k+1}{k}.
  \end{equation}
\end{claim}
\begin{proof}
  Consider a tube with $n$ particles, $k$ of them are X, $n-k$ are Y.
  \begin{equation*}
    \underbrace{\text{YY}\dots\text{YY}}_{n-k}
  \end{equation*}
  Placing all $k$ of the X particles in the $n-k+1$ gaps will ensure no consecutive X's.
\end{proof}

\end{frame}

\end{document}
