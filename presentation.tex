\documentclass[xcolor={usenames,dvipsnames}]{beamer}

\usetheme{Dresden}
\usecolortheme{beaver}

\usepackage{import}
\usepackage{xifthen}
\usepackage{pdfpages}
\usepackage{transparent}
\usepackage{caption}

\newcommand{\incfig}[1]{%
  \def\svgwidth{0.3\textwidth}
  \import{./figures/}{#1.pdf_tex}
}


% https://tex.stackexchange.com/questions/137022/how-to-insert-page-number-in-beamer-navigation-symbols
\addtobeamertemplate{navigation symbols}{}{%
  \usebeamerfont{footline}%
  \usebeamercolor[fg]{footline}%
  \hspace{1em}%
  \insertframenumber/\inserttotalframenumber
}
\setbeamercolor{footline}{fg=darkred}
\usepackage{mathtools, amssymb, amsthm, parskip}

\usepackage[inline]{asymptote}
\def\asydir{asy}

\usepackage{microtype}
\usepackage{multicol}

\newtheorem{claim}{Claim}[theorem]

\title{Question 8}
\subtitle{Mathematics and Statistics Research Competition}
\author{Jiamu Li \& Frank Tang \& Edward Wang}
\institute{Scotch College}
\date{October 24, 2022}

\DeclareMathOperator{\prob}{Pr}
\DeclarePairedDelimiter\floor{\lfloor}{\rfloor}

\begin{document}

\begin{frame}
  \titlepage
\end{frame}


\section{The Question}
\begin{frame}{The Situation}
  A particle generator emits X or Y particles into an empty tube, with equal probability. Shots are independent.
  \begin{center}
    
  \end{center}
\end{frame}
\section{Problem 1}
\begin{frame}{Problem 1}
\begin{itemize}
  \item What is the probability that no two X-particles are next to each other?
\end{itemize}

\begin{equation*}
  \prob(\text{No consec X-particles}) = \frac{\# \text{Arrangements w/o consec X-particles}}{\# \text{Total arrangements}}
\end{equation*}

\end{frame}

\begin{frame}

  Consider a tube with $n$ particles, $k$ of them are X, $n-k$ are Y.
  \begin{equation*}
    \underbrace{\text{YY}\dots\text{YY}}_{n-k}
  \end{equation*}
  Placing all $k$ of the X particles in the $n-k+1$ gaps will ensure no consecutive X's. Hence, the number of arrangements without consecutive X's is
  \begin{equation*}
    \sum_{k=0}^n \binom{n-k+1}{k}.
  \end{equation*}

\end{frame}

\begin{frame}[fragile]

\begin{multicols}{2}
  \begin{center}
    \begin{asy}
      unitsize(0.38cm);
      defaultpen(fontsize(9pt));

      int fib(int n) {
        if (n == 0) 
        return 0;
        else if (n == 1) 
        return 1;
        int fibs[] = {0, 1};
        while (fibs.length <= n) {
          fibs.push(fibs[fibs.length - 1] + fibs[fibs.length - 2]);
        }
        return fibs[fibs.length - 1];
      }

      int rows = 6;

      for (int i = 0; i <= rows; ++i) {
        pair A = (-i, -i*sqrt(3));
        pair B = (3, sqrt(3));
        path idk = A -- (A + 0.5*i*B) + 0.5*B;
        draw(L = Label(string(fib(i+1)), position = EndPoint), idk, blue+opacity(0.4));
      }

      for (int i = 0; i < floor(rows/2)+1; ++i) {
        pair A = (-rows, -rows*sqrt(3));
        pair B = (A + i*(3, sqrt(3)));
        if (i != 0)
        draw(B + 0.25*(-1, sqrt(3)) -- B + 0.75*(-1, sqrt(3)), red);
        if (i != floor(rows/2))
        draw(B + 0.25*(1, sqrt(3)) -- B + 0.75*(1, sqrt(3)), red);
      }

      for (int i = 0; i <= rows; ++i) {
        for (int j = 0; j <= i; ++j) {
          label(string(choose(i, j)), (j*2 - i, -i*sqrt(3)));
        }
      }
    \end{asy}
  \end{center}
  \columnbreak
  Observe the diagram of Pascal's triangle.

  The formula represents the sum of each diagonal, which is made up of the sum of the previous two diagonals, satisfying the Fibonacci recursion. 

  Hence,   \[\sum_{k=0}^n \binom{n-k+1}{k} = F_{n+2}.\]
\end{multicols}

\end{frame}

\begin{frame}
  Since each of the $n$ particles is X or Y, the number of arrangements is $2^n$.

  Hence the probability that no two X-particles are consecutive after $n$ shots is \[
    \frac{F_{n+2}}{2^n}
  .\] 
\end{frame}

\section{Problem 2}
\begin{frame}{Problem 2}
  Two consecutive X particles now collapse into one. 
  \begin{itemize}
    \item Find the average number of particles after $n$ shots.
  \end{itemize}
  \begin{multicols}{2}
  \begin{figure}[ht]
    \centering
    \incfig{something}
    \caption{}
    \label{fig:stuff}
  \end{figure}
  \columnbreak
  Out of 4 possible events, 3 increase the number of particles by 1. Hence, the average number of particles increases by $\frac{3}{4}$ each shot, so the formula is \[
    T_n = \frac{3}{4}n + \frac{1}{4}
  .\] 
  \end{multicols}
\end{frame}

\section{Problem 3}
\begin{frame}{Problem 3}
  Continuing from the previous problem, let the probability of firing an X particle be some  $p \in (0, 1)$.
  \begin{itemize}
    \item What is the proportion of X particles in the tube as the number of shots goes to infinity?
  \end{itemize}
  \vspace{1cm}
  \begin{equation*}
    \text{Proportion} = \frac{\#\text{X particles}}{\#\text{Particles}} = \frac{\#\text{Particles - Y particles}}{\#\text{Particles}}
  \end{equation*}
\end{frame}

\begin{frame}
  The number of X particles stays the same if an X particle hits an X particle.

  This happens with probability $p^2$.

  Hence the average number of particles increases by $1-p^2$ per shot, so after $n$ shots,  \[
    T_n = (1-p^2)n + p^2
  .\] 
\end{frame}
\begin{frame}
  The average number of Y particles is clearly \[
    (1-p)n
  .\] Hence, the proportion is 
  \begin{equation*}
    \text{Proportion} = \frac{(1-p^2)n + p^2 - (1-p)n}{(1-p^2)n + p^2}.
\end{equation*}
As $n$ approaches infinity, the proportion approaches \[
  \frac{p}{1+p}
.\] 
\end{frame}

\section{Generalisation}
\begin{frame}{Generalisation}
  \begin{itemize}
    \item What happens if $m$ consecutive X particles collapsed into $n$ particles?
    \item Everything else remains the same, i.e. Y particles don't collapse, probability of X particle is $p \in (0, 1)$
    \item What is the average number of particles after $k$ shots?
  \end{itemize}
\end{frame}

\begin{frame}{Recursion}
  If an X particles hits $m-1$ consecutive X particles, then they collapse into $n$ X particles, decreasing the number of particles by $m-1-n$.

  Otherwise, the number of particles increases by 1.

  Let the probability of having  $m-1$ consecutive X particles be $\vartheta$. 

  \begin{align*}
    T_{k+1} &= T_k + 1-p\vartheta + p\vartheta(n-m+1) \\
            &= T_k + 1 - p\vartheta(m - n)
  \end{align*}
\end{frame}
\begin{frame}{Calculating $\vartheta$}
  $\vartheta$ is the probability that there are $m-1$ X particles in a row.

  The string of X particles must start with either a Y unless it is the beginning of the sequence.
  \begin{figure}[H]
    \vspace{-1.5em}
    \begin{align*}
        &\underbrace{\text{X}\dots\text{X}}_{m-1}\text{Y} \tag{A}\\
        &\underbrace{\text{X}\dots\text{X}}_{m-1} \tag{B}
    \end{align*}
    \vspace{-2em}
  \end{figure}
\end{frame}
\begin{frame}{Configuration A}
    \begin{figure}[H]
    \vspace{-1.5em}
    \begin{align*}
        &\underbrace{\text{X}\dots\text{X}}_{m-1}\text{Y} \tag{A}
    \end{align*}
    \vspace{-2em}
  \end{figure}

  If the sequence has a Y particle at the end, then the probability is simply $(1-p)p^{m-1}$.

  But that doesn't account for previous collapses.
  \begin{figure}[H]
      \vspace{-1.5em}
      \begin{equation*}
        \underbrace{\text{X}\dots\text{X}}_{m + (m-n) - 1}\text{Y} \quad \longrightarrow \quad \underbrace{\text{X}\dots\text{X}}_{n+(m-n)-1}\text{Y}\quad = \quad \underbrace{\text{X}\dots\text{X}}_{m-1}\text{Y} 
      \end{equation*}
      \vspace{-2em}
    \end{figure}
  \end{frame}
  \begin{frame}{Configuration A}
    Adding $m-n$ particles to $m-1$ reverts it back to $m-1$, which can happen $\floor*{\dfrac{k-m}{m-n}}$ times. Summing this up gives \[\sum_{a = 0}^{\floor*{\frac{k-m}{m-n}}} (1-p)p^{a(m-n) + m-1}.\]
\end{frame}
\begin{frame}{Configuration B}
    \begin{figure}[H]
    \vspace{-1.5em}
    \begin{align*}
        &\underbrace{\text{X}\dots\text{X}}_{m-1} \tag{B}
    \end{align*}
    \vspace{-2em}
  \end{figure}
  Accounting for previous collapses, this configuration can obviously happen with a probability of $p^{k}$, if $k = a(m-n) + m-1$ for some integer $a$. Hence, the probability is \[
    \varepsilon = \begin{cases} p^{k}, &k-m+1 \equiv 0 \bmod m-n \\ 0, &k-m+1 \not\equiv 0 \bmod m-n. \end{cases}
  .\] 
\end{frame}
\begin{frame}
  Adding these two probabilities gives us \[
    \vartheta = \sum_{a=0}^{\floor*{\frac{k-m}{m-n}}} (1-p)p^{a(m-n)+m-1} + \varepsilon,
  \] where \[
    \varepsilon = \begin{cases} p^{k}, &k-m+1 \equiv 0 \bmod m-n \\ 0, &k-m+1 \not\equiv 0 \bmod m-n. \end{cases}
  .\] Plugging this into the recursion earlier, we get \[
    T_k = m-1+\sum _{b=m-1}^{k-1} \left(1-p \left(\sum _{a=0}^{\floor*{\frac{b-m}{m-n}} } (1-p) p^{a (m-n)+m-1}+\varepsilon\right) (m-n)\right)
  .\] 
\end{frame}

\begin{frame}{Use of computer simulation}
  Computer models allowed us to quickly compute average number of particles after any number of shots.

  These helped us determine that a formula existed when finding the pattern.

  They also helped us validate that our formula worked correctly.
\end{frame}

\end{document}
